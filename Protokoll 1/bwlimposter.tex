% tudbwlim-Vorlage zur Erstellung eines Posters in der Version 2.1
%
%
% Beispieldatei zur Benutzung der IM-Vorlage. Für diese Vorlage wird das Paket "tudscr" von Falk Hanisch (http://wwwpub.zih.tu-dresden.de/~fahan/tudscr/) in der Version >= 2.05m benötigt. Ist es nicht vorhanden, kann es über den TeXLive Manager oder den Kommandozeilenaufruf "tlmgr update tudscr" aktualisiert werden.
% Das Paket scrbase sollte in der Version >= 3.18 installiert sein.

% Weitere Informationen über das Corporate Design der TU Dresden und die bereitgestellten Vorlagen finden sich hier: http://tu-dresden.de/service/publizieren/cd/4_latex, sowie Details zur Verwendung der Vorlage (in Kommandozeile "texdox tudscr" oder http://ftp.uni-erlangen.de/ctan/macros/latex/contrib/tudscr/doc/tudscr.pdf), als auch nützliche Tipps zur Verwendung von \LaTeX und zum Schreiben von wissenschaftlichen Arbeiten allgemein (https://github.com/tud-cd/tudscr/releases -> tudscr_v2.xx.zip -> .\doc\latex\tudscr\tutorials\treatise.pdf). Ebenfalls bereitgestellt von Falk Hanisch.



\RequirePackage{hyphsubst} % Zur verbesserten Worttrennung
% Bitte an die Sprache anpassen die an \documentclass übergeben wird
	\HyphSubstLet{ngerman}{ngerman-x-latest}
	%\HyphSubstLet{english}{usenglishmax}
\RequirePackage{fix-cm}

\documentclass[english, ngerman, a0paper, fontsize=30pt,
cdfont=false,cdfont=nodin,cdmath=false,
cdgeometry=custom,
bleedmargin=3mm,
backgroundcolor=white,
]{./settings/tudscrposter}

\geometry{bottom=12cm, textwidth=0.85\paperwidth}


\usepackage{settings/tudbwlimPackages}
\usepackage{settings/tudbwlimStyle}

\usepackage{tikz,pgfplots}
	\pgfplotsset{compat=newest}
\usepackage{multicol}
\usepackage{tcolorbox}
	\tcbuselibrary{breakable}
	\tcbuselibrary{skins}
	% Hinweis  ##########
	%####################
	% Das Paket "tcolorbox" bietet viele Möglichkeiten Inhaltsboxen zu definieren und das Layout anzupassen (siehe die raster bzw. poster library).
	%\tcbuselibrary{raster} % tcolorbox.pdf pages 276 -- 293
	% Poster layout library: p. 395 -- 407
	%\tcbuselibrary{poster} % poster tutorial: http://ctan.sharelatex.com/tex-archive/macros/latex/contrib/tcolorbox/tcolorbox-tutorial-poster.pdf
	%####################
	% Eine andere Möglichkeit die hier nicht aufgezeigt ist, ist mit tikz oder pstrickts das Layout/die Position der Textinhalte manuell zu setzen. Diese Möglichkeit ist hier nicht aufgezeigt.
\usepackage{varwidth}
\usepackage{blindtext}

\usepackage{wrapfig}


\usepackage[backend=biber,bibencoding=auto,citestyle=authoryear-ibid,bibstyle=authoryear,natbib=true,]{biblatex}
	\usepackage{settings/BiblatexSetup}
\usepackage[automake=false,acronym,symbols,nomain,translate=babel,]{glossaries}
	\usepackage{settings/GlossariesSetup}

\usepackage[default]{opensans}

%################ Eigene Einstellungen/Befehle #######################


%################ Sonstige Einstellungen/Befehle #####################



%################ Abkürzungen #######################
%\makeglossaries
\glsenableentrycount % aktiviert \cgls, \cglspl, \cGls, \cGlspl, siehe https://tex.stackexchange.com/questions/98494/glossaries-dont-print-single-occurences/230664#230664

\renewcaptionname{ngerman}{\contactname}{Impressum}
\renewcaptionname{english}{\contactname}{Impressum}

%\RequirePackage[breaklinks=true, colorlinks=false, hidelinks]{hyperref} % frenchlinks=true,
%	\hypersetup{pdfprintscaling=None} % gleiches Verhalten, auch ohne hyperref, liefert: \pdfcatalog{/ViewerPreferences<</PrintScaling/None>>}



%%% Definition von Inhaltsboxen, vgl. die tcolorbox Dokumentation. Bei der Gestaltung des Posters bitte auf eine Art von Boxen beschränken!
\newtcolorbox[]{bwlimbox}[2][]{%
colback=white, colframe=cdblue!20, coltitle=cddarkblue, %colbacktitle=cdblue!25, 
drop small lifted shadow,
%title filled, 
boxrule=0pt,
%titlerule=10mm,titlerule style=red,
arc=1mm, sharp corners=south,
fonttitle=\usefontofkomafont{section}, 
title=\ifdin{\MakeTextUppercase{#2}}{\MakeTextUppercase{#2}}, #1
}

\newtcolorbox[]{testbox}[2][]{%
	colback=cdblue!05, colframe=cdblue!20, coltitle=cddarkblue,
	drop small lifted shadow,
	enhanced, attach boxed title to top center={yshift=-\tcboxedtitleheight/3,yshifttext=-6mm},boxed title size=copy,boxed title style={colframe=cdblue!20,colback=cdblue!20},
	fonttitle=\usefontofkomafont{section}, 
	title=\ifdin{\MakeTextUppercase{#2}}{\MakeTextUppercase{#2}}, #1
}

\newtcolorbox[]{testboxtwo}[2][]{%
	enhanced,before skip=2mm,after skip=2mm,
	colback=cdblue!05,colbacktitle=cdblue!40,colframe=cdblue!20,coltitle=cddarkblue,boxrule=0.2mm,
	attach boxed title to top center={xshift=1cm,yshift*=1mm-\tcboxedtitleheight},
	varwidth boxed title*=-3cm,
	drop small lifted shadow,
	boxed title style={frame code={
			\path[fill=tcbcol@back!30!black]
			([yshift=-1mm,xshift=-1mm]frame.north west)
			arc[start angle=0,end angle=180,radius=1mm]
			([yshift=-1mm,xshift=1mm]frame.north east)
			arc[start angle=180,end angle=0,radius=1mm];
			\path[fill=cdblue!20] %left color=tcbcol@back!60!black,right color=tcbcol@back!60!black,middle color=tcbcol@back!60!black
			([xshift=-2mm]frame.north west) -- ([xshift=2mm]frame.north east)
			[rounded corners=1mm]-- ([xshift=1mm,yshift=-1mm]frame.north east)
			-- (frame.south east) -- (frame.south west)
			-- ([xshift=-1mm,yshift=-1mm]frame.north west)
			[sharp corners]-- cycle;
		},interior engine=empty,
	},
	fonttitle=\usefontofkomafont{section},
	title=\ifdin{\MakeTextUppercase{#2}}{\MakeTextUppercase{#2}}, #1
}

\newtcolorbox
{mybox}[2][]{enhanced,skin=enhancedlast jigsaw,
	attach boxed title to top left={xshift=-4mm,yshift=-0.5mm},
	varwidth boxed title=1.2\linewidth,
	colback=cdblue!05, colbacktitle=cdblue!20, colframe=cdblue!20, coltitle=cddarkblue,
	boxed title style={empty,arc=0pt,outer arc=0pt,boxrule=0pt},
	drop small lifted shadow,
	underlay boxed title={
		\fill[cdblue!20] (title.north west) -- (title.north east)
		-- +(
		\tcboxedtitleheight
		-1mm,-
		\tcboxedtitleheight
		+1mm)
		-- ([xshift=4mm,yshift=0.5mm]frame.north east) -- +(0mm,-1mm)
		-- (title.south west) -- cycle;
		\fill[cdblue!45!white!50!black] ([yshift=-0.5mm]frame.north west)
		-- +(-0.4,0) -- +(0,-0.3) -- cycle;
		\fill[cdblue!45!white!50!black] ([yshift=-0.5mm]frame.north east)
		-- +(0,-0.3) -- +(0.4,0) -- cycle;  },
	fonttitle=\usefontofkomafont{section},
	title=\ifdin{\MakeTextUppercase{#2}}{\MakeTextUppercase{#2}},#1}

% Definition der Fußzeile, siehe hierzu die tudscr-Doku auf Seiten 32, 58--62:
\makeatletter
\footcontent[\small]%
{
	\begin{minipage}{0.09\paperwidth} % Schafft Platz für das Lehrstuhllogo in der Fußzeile.
		\hfill
	\end{minipage}
	\begin{minipage}[t][][t]{0.21\paperwidth}
		{\bfseries\contactname}\newline
		Technische Universität Dresden\newline
		\@faculty\newline
		Lehrstuhl für Betriebswirtschaftslehre,\newline
		inbes. Industrielles Management\newline
		Prof. Dr. Udo Buscher
	\end{minipage}
	\begin{minipage}[t][][t]{0.19\paperwidth}
		{\bfseries Diplomarbeit von}\newline
		Vorname Nachname\newline
		vorname.nachname@mailservice.de\newline
		{\bfseries\contactpersonname}\newline
		Max Mustermann\newline
		max.mustermann@tu-dresden.de
	\end{minipage}
	\begin{minipage}[t][][t]{0.19\paperwidth}
		% hier ist Platz für weitere Betreuer, einen Industriepartner, Förderverweis etc...
	\end{minipage}
}
\makeatother









\begin{document}
\microtypesetup{protrusion=true}
\spacing{1.0}
\sffamily

\title{\MakeTextUppercase{Ein schöner Titel muss hier her -- und wenn er lang ist, dann auch über zwei Zeilen und linksbündig}}
\subtitle{\MakeTextUppercase{Optionaler Unter-/Zweittitel}}

%%% IM
\footlogo{./settings/IM-Logo}
\chair{Lehrstuhl für BWL, insbes. Industrielles Management, Prof. Dr. Udo Buscher}

%%% CBM
%\footlogo{./settings/CBM-Logo}
%\chair{Lehrstuhl für BWL, insbes. Industrielles Management -- Zentrum Car Business Management}

\maketitle

%################ Abstract (wenn nötig) ######################

% Optionaler Abstract.
%\begin{abstract}%[columns=2]
%	\noindent
%%	\begin{bwlimbox}[notitle]{}
%		\blindtext
%%	\end{bwlimbox}
%\end{abstract}
%\bigskip



\noindent
\begin{minipage}{0.48\textwidth}
	\begin{mybox}[equal height group=A]{Thema der Arbeit}
		% equal height group enables that two boxes have the same height.
		% [tcolorbox.pdf] p. 61: Note that you have to compile twice to see changes in the height groups.
		\blindtext
	\end{mybox}
\end{minipage}
\hfill
\begin{minipage}{0.48\textwidth}
	\begin{mybox}[equal height group=A]{Motivation}
		\blindtext[2]
	\end{mybox}
\end{minipage}



\bigskip



\begin{testboxtwo}{Grundlagen}
	\begin{multicols}{2}%[\section*{Test}]
		\blindmathtrue
		\blindtext
		\begin{center}
			\begin{tikzpicture}%[scale=0.975]
			\begin{axis}[
			xmin=-0.5, xmax=5,
			ymin=-0.5, ymax=5,
			xtick={1,...,4},
			ytick={1,...,4},
			axis y line=center,
			axis x line=middle,
			ylabel={$y$},
			xlabel={$x$},
			xlabel style={below right},
			ylabel style={above left},
			]
			
			\node (A) at (axis cs:2,1) {};
			\node[right] at (A) {$A$};
			\draw[fill] (A) circle [radius=1.75pt];
			
			\end{axis}
			\end{tikzpicture}
			\begin{tikzpicture}%[scale=0.975]
			\begin{axis}[
			xmin=-0.5, xmax=5,
			ymin=-0.5, ymax=5,
			xtick={1,...,4},
			ytick={1,...,4},
			axis y line=center,
			axis x line=middle,
			ylabel={$y$},
			xlabel={$x$},
			xlabel style={below right},
			ylabel style={above left},
			]
			
			\node (B) at (axis cs:2,1) {};
			\node[right] at (B) {$B$};
			\draw[fill] (B) circle [radius=1.75pt];
			
			\end{axis}
			\end{tikzpicture}
			\begin{tikzpicture}%[scale=0.975]
			\begin{axis}[
			xmin=-0.5, xmax=5,
			ymin=-0.5, ymax=5,
			xtick={1,...,4},
			ytick={1,...,4},
			axis y line=center,
			axis x line=middle,
			ylabel={$y$},
			xlabel={$x$},
			xlabel style={below right},
			ylabel style={above left},
			]
			
			\node (C) at (axis cs:2,1) {};
			\node[right] at (C) {$C$};
			\draw[fill] (C) circle [radius=1.75pt];
			
			\end{axis}
			\end{tikzpicture}
		\end{center}
	\end{multicols}
\end{testboxtwo}



\bigskip



%\begin{testbox}{Beispieltitel}
%	\blindtext[2]
%\end{testbox}


\begin{testbox}{Rechenergebnisse}
	\begin{wrapfigure}[10]{l}[0pt]{0mm} % Vergleiche floatrow-Doku Seite 82.
		% Notes. 1) For figure, contents in e.g. in wrapfigure environment you set width in mandatory argument. If you’ll write 0mm as {<width of figure>} argument, the wrapfig package will calculate a natural width of float contents.
		\tcbincludegraphics[colback=cdblue!05, colframe=cdblue!05, boxrule=0pt, width=0.4\textwidth]{TUD-black}
	\end{wrapfigure}
	\blindtext[2]
\end{testbox}



\bigskip



\noindent
\begin{minipage}{0.25\textwidth}
	\begin{mybox}{Symbole}
		\begin{tabular}{@{}ll@{\quad}ll@{}}
			$a$ & Ausprägung 1 & $b$ & Ausprägung 2\\
			$c$ & Ausprägung 3 & &
		\end{tabular}
	\end{mybox}
\end{minipage}


\end{document}