\RequirePackage[cdmath=false,cdfont=false]{fix-tudscrfonts}
\makeatletter
\def\input@path{{./settings/}}
\makeatother

\documentclass
[
german,
nosectionnum,
cdmath=false, cdfont=false, % Serifen Schriften für Matheumgebung
]{tudbeamer} % ,handout,draft

\usepackage{selinput}\SelectInputMappings{adieresis={ä},germandbls={ß}}
\usepackage[T1]{fontenc}
\usepackage{lmodern}

\usepackage{enumitem}
\usepackage{varwidth}

\usepackage[backend=biber,bibencoding=auto,citestyle=authoryear-ibid,bibstyle=authoryear,natbib=true,]{biblatex}
	\usepackage{settings/BiblatexSetup}


\usepackage[default,scale=0.95]{opensans}

%################ Bibliographie laden #######################
\addbibresource{./settings/Quellen.bib} % Pfad/Name der .bib-Datei

\begin{document}
	
\sffamily % Serifenlose Schriften für die Folien

\einrichtung{Fakultät Wirtschaftswissenschaften} % Faculty of Business and Economics
\institut{Lehrstuhl für BWL, insbes. Industrielles Management, \mbox{Prof. Dr. Udo Buscher}}% Chair of Business Management especially Industrial Management

\title[Kurztitel für Fußzeile]{Titel der Präsentation}
\subtitle{Optionaler Untertitel}

\author{Vorname Nachname}
\datecity{Dresden, \today}

\maketitle
	
	
	
% Gliederung am Beginn jeder Section
\AtBeginSection[]
{	
	{
	\setbeamertemplate{headline}%
	{%
		\vskip6.15mm\color{cdgray}
		\setlength{\arrayrulewidth}{0.1pt}
	}
	
	\setbeamertemplate{footline}{}%
	\setbeamerfont{section in toc}{size=\small}
	\setbeamerfont{subsection in toc}{size=\scriptsize}
		
	\begin{frame}{Gliederung}
	
	\renewcommand{\baselinestretch}{1.9}
	\tableofcontents[sectionstyle=show/shaded,subsectionstyle=show/shaded/hide,subsubsectionstyle=show/show/shaded/hide]
	\addtocounter{framenumber}{-1}
	\end{frame}
	}
}
	
	
% Beginn Dokument
	
	
	
\section{Erster Abschnitt}
\begin{frame}[t]{Test}
	Inhalt
\end{frame}



\section{Zweiter Abschnitt}
\begin{frame}{Test}
	Inhalt
	\begin{figure}
		\includegraphics[width=0.1\textwidth]{settings/IM-Logo}
	\end{figure}
\end{frame}



\section{Dritter Abschnitt}
\begin{frame}{Test}
	\begin{align*}
		a^2 + b^2 = c^2
	\end{align*}
\end{frame}



\section{Vierter Abschnitt}
\begin{frame}{Test}
	\begin{itemize}[label=\EnumerationSquare]
		\item Test 1
		\item Test 2
		\item Test 3
		\begin{itemize}[label=\EnumerationDash]
			\item Test 1
		\end{itemize}
	\end{itemize}
\end{frame}


\appendix
\begin{ClosingFrame}{}
	\Large\bfseries
	\begin{center}
		Vielen Dank für Ihre Aufmerksamkeit!
	\end{center} 
\end{ClosingFrame}


\section*{Backup}
\begin{BackupFrame}{Backupfolie}
	content...
\end{BackupFrame}

	
\end{document}